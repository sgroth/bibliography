% !TEX TS-program = pdflatexmk
\documentclass[]{article}

\RequirePackage[l2tabu, orthodox]{nag}

\usepackage[ngerman]{babel} 
\usepackage[T1]{fontenc}
\usepackage[utf8]{inputenc}

% Fonts and Typography
\usepackage[osf]{ebgaramond} % Garamond for rm
\usepackage[osf]{sourcesanspro} % SourceSansPro for ss
\usepackage[scale=1]{sourcecodepro} % SourceCodePro for tt
\usepackage{microtype}
\usepackage[autostyle=true,german=quotes]{csquotes}
\usepackage[colorlinks, urlcolor=magenta]{hyperref}
\renewcommand\UrlFont{\color{magenta}\rmfamily}

% Version Control
\usepackage{gitfile-info}

% Headers and Footers
\usepackage{fancyhdr}
\pagestyle{fancy}
\renewcommand{\headrulewidth}{0pt}
\fancyhead{}
\fancyfoot{}
\rfoot{\thepage}
\lfoot{\textcolor{gray}{Version: \gfiGetCommitAbr, \gfiGetYear--\gfiGetMonth--\gfiGetDay, \footnotesize{\texttt{\href{https://github.com/sgroth/bibliography}{https://github.com/sgroth/bibliography}}}}} % adds version control information

% Bibliography
\usepackage[authordate,notes,backend=biber,isbn=false,annotation,firstinits=true,defernumbers=true,sorting=ydnt]{biblatex-chicago}
\addbibresource{sgroth-bibliography.de.bib}
\DeclareFieldFormat{labelnumberwidth}{}
\setlength{\biblabelsep}{0pt}
\DeclareFieldFormat{annotation}{{#1}}

\begin{document}

\title{Schriftenverzeichnis}
\author{Dr. Stefan Groth\\\footnotesize{\texttt{\href{https://www.stefangroth.com}{https://www.stefangroth.com}}}}
\date{Version: \href{https://github.com/sgroth/bibliography}{\gfiGetCommitAbr}, \gfiGetYear--\gfiGetMonth--\gfiGetDay}

\maketitle

\nocite{*} % includes all entries from my personal bibliography

% Print segmented bibliographies
\printbibliography[keyword=submitted,title={Eingereichte Beitr\"age}]
\printbibliography[type=unpublished,keyword=inpreparation,title={Beitr\"age in Vorbereitung}]
\printbibliography[type=book,notkeyword=submitted,notkeyword=inpreparation,keyword=monography,title={Monographien}]
\printbibliography[type=book,notkeyword=submitted,notkeyword=inpreparation,keyword=editedvolume,title={Herausgeberschaften}]
\printbibliography[type=article,notkeyword=submitted,notkeyword=inpreparation,title={Zeitschriftenaufs\"atze}]
\printbibliography[type=incollection,notkeyword=submitted,notkeyword=inpreparation,title={Aufs\"atze in Sammelb\"anden}]
\printbibliography[type=review,notkeyword=submitted,notkeyword=inpreparation,title={Rezensionen}]

\end{document}
